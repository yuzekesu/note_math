\section{Kapitel 1} 
\subsection{\textbf{axiom}}\label{axiom} 
    \textbf{\cite{axiom}}Påstående om \textbf{grundläggande begreppen} som man accepterar som sanna eller korrekta, \textbf{är ett självklart sant påstående som antas vara sant utan bevis.}
\subsection{\textbf{parallellaxiom}}\label{parallellaxiom} 
    See \textbf{axiom} i [\ref{axiom}].
    \\\textbf{Exempel:}
    \vspace{3 mm}
    \begin{adjustwidth}{10 mm}{10 mm}
        Om \textsl{L} är en rät linje och \textsl{P} är en punkt som inte ligger på \textsl{L}, så finns det en och endast en rät linje genom \textsl{P} som är parallell med \textsl{L}, d.v.s som inte skär \textsl{L}.
    \end{adjustwidth}
    \vspace{3 mm}
\subsection{\textbf{härledningsregel}}\label{härledningsregel}
    \textbf{\cite{härledningsregel}}Det finns ett antal tillåtna logiska regler - s.k. \textbf{härledningsregler} - som man kan använda för att härleda ett nytt sant påstående \textbf{från axiomen}[\ref{axiom}].
    \\\textbf{Exempel:}
    \vspace{3 mm}
    \begin{adjustwidth}{10 mm}{10 mm}
        \textbf{Regel 1.} Om vi vet att ett påstående \textsl{p} medför ett påstående \textsl{q} och vi kan visa att \textsl{p} är sant, så är också \textsl{q} sant.
        \\\textbf{Regel 2.} Om vi vet att ett påstående \textsl{p} medför ett annat påstående \textsl{q} och att \textsl{q} medför ett tredje påstående \textsl{r}, kan vi dra slutsatsen att \textsl{p} medför \textsl{r}.
    \end{adjustwidth}
    \vspace{3 mm}
\subsection{\textbf{sats \& teorem}}\label{sats}
    \textbf{\cite{sats}}Med hjälp av härledningsregler[\ref{härledningsregel}] skapar vi nya nåstående som kallas för \textbf{bevis} och det nya sanna påståendet kallas för en \textbf{sats} (eller ett \textbf{teorem}). Satserna kan sedan i sin tur utnyttjas som delar i bevis av ytterligare satser.
    \\\textbf{En matematisk sats kan ofta formuleras på förljande sätt:}
    \vspace{3 mm}
    \begin{adjustwidth}{10 mm}{10 mm}
        Om påståendet \textsl{p} är sant, så är också påstående \textsl{q} sant.
    \end{adjustwidth}
    \vspace{3 mm}
    I det ovanståend exemplet. Påstående \textsl{p} kallas \textbf{satsens förutsättning} och påståendet \textsl{q} kallas \textbf{satsens slutsats}(Observera att det är möjligt att en sats har flera förutsättningar). Man säger också att \textsl{q} \textbf{följer} av \textsl{p}, och att \textsl{p} \textbf{medför} \textsl{q}.
    \\\textbf{En annan exempel:}
    \vspace{3 mm}
    \begin{adjustwidth}{10 mm}{10 mm}
        Om en triangel med sidorna \textsl{a}, \textsl{b} och \textsl{c} har en rät vinkel mellan sidorna \textsl{a} och \textsl{b}, så är $a^2+b^2=c^2$.
    \end{adjustwidth}
    \vspace{3 mm}
\subsection{\textbf{hjälpsats}}\label{hjälpsats}
    See \textbf{sats} i [\ref{sats}]. En enkel \textbf{sats} som används för att bevisa en mer komplicerad sats kallar man för en \textbf{hjälpsats} (eller ett \textbf{lemma}).
\subsection{\textbf{följdsats}}\label{följdsats}
    See \textbf{sats} i [\ref{sats}]. En \textbf{sats} som enkelt kan bevisas med hjälp av en annan sats kallas för en \textbf{följdsats} (eller ett \textbf{korollarium}).
\subsection{{\textbf{förmodan}}}\label{förmodan}
    \textbf{\cite{förmodan}} För att förmulera nya satser studerar man ofta olika specialfall och försöker med hjälp av dessa få en inblick i allmänna företeelser. Sedan formulerar man ett påstående (en gissning) som man inte vet om det är sant eller falskt. Ett sådant påstående kallas för en \textbf{förmodan} (eller en \textbf{hypotes}). Ibland kan det ta lång tid att avgöa om en förmodan är sann eller inte. Det görs genom att antingen bevisa den eller ge ett motexempel som visar att den är falsk.

